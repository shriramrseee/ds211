\documentclass[12pt,letterpaper]{article}
%\usepackage{fullpage}
\usepackage[top=2cm, bottom=4.5cm, left=2.5cm, right=2.5cm]{geometry}
\usepackage{amsmath,amsthm,amsfonts,amssymb,amscd}
%\usepackage{lastpage}
\usepackage{enumerate}
\usepackage{fancyhdr}
%\usepackage{mathrsfs}
\usepackage{xcolor}
\usepackage{graphicx}
\usepackage{listings}
\usepackage{hyperref}
\usepackage{tikz}

\hypersetup{%
  colorlinks=true,
  linkcolor=blue,
  linkbordercolor={0 0 1}
}
 
\renewcommand\lstlistingname{Algorithm}
\renewcommand\lstlistlistingname{Algorithms}
\def\lstlistingautorefname{Alg.}

\lstdefinestyle{Python}{
    language        = Python,
    frame           = lines, 
    basicstyle      = \footnotesize,
    keywordstyle    = \color{blue},
    stringstyle     = \color{green},
    commentstyle    = \color{red}\ttfamily
}

\setlength{\parindent}{0.0in}
\setlength{\parskip}{0.05in}

% Edit these as appropriate

\pagestyle{fancyplain}
\headheight 35pt
\lhead{DS 211 \\ Numerical Optimization}                 % <-- Comment this line out for problem sets (make sure you are person #1)
\chead{\textbf{\Large Lecture 17}}
\rhead{Scribe: Shriram R.}
\lfoot{}
\cfoot{}
\rfoot{\small\thepage}
\headsep 1.5em

\newcommand{\R}{\mathbb{R}}


\begin{document}

\section{Standard Form}

The standard form of a linear programming problem is as follows,

\begin{equation*}
\min c^{T}x, \text{ subject to } Ax = b, x \geq 0,
\end{equation*}

where $A \in \R^{mxn}$, $c$ and $x$ $\in \R^{nx1}$ and $b \in \R^{mx1}$. Also, $m \leq n$ and Rank$(A) = m$. Often, framing a linear programming problem in the standard form is an important task itself.

\section{Revised Simplex Algorithm}

\begin{center}
	\begin{tikzpicture}
	\draw [->] (5,-1) -- (5,5);
    \draw [->] (4,0) -- (11,0);
	\draw [] (4,4) -- (10,1);
	\draw [] (8,-1) -- (8,5);
	\node[] at (9,4) {$x_1 = 1$};
	\node[] at (10,2) {$x_1 + x_2 = 2$};
	\node[] at (8,-0.1) {*};
	\node[] at (8, 1.9) {*};
	\node[] at (5, -0.1) {*};
	\node[] at (5, 3.4) {*};
	\node[] at (4,3) {\small{a:$(0,2,1,0)$}};
	\node[] at (4,-0.5) {\small{b:$(0,0,1,2)$}};
	\node[] at (9,-0.5) {\small{c:$(1,0,0,1)$}};
	\node[] at (7,1.7) {\small{d:$(1,1,0,0)$}};
	\end{tikzpicture}
\end{center}



\end{document}